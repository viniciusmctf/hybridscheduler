\documentclass[12pt]{article}

\usepackage{sbc-template}
\usepackage{graphicx,url}
\usepackage{mathtools}
\usepackage[utf8]{inputenc}  
\usepackage[brazil]{babel}
\usepackage[portuguese, onelanguage, ruled, linesnumbered]{algorithm2e}
\usepackage{amssymb,amsmath,amsthm,amssymb}

\newcommand{\lb}{\textit{PseudoDispersalLB}}

\sloppy

\title{Escalonador em Hardware com Interface Avalon}
% A core-guided load-balancing algorithm prototype: PseudoDispersalLB

\author{Luiz Henrique de Lorenzi Cancellier, Vinicius Marino Calvo Torres de Freitas}

\address{Departamento de Informática e Estatística -- Universidade Federal de
Santa Catarina\\
Florianópolis -- SC -- Brasil
\email{luizhenriquecancellier@gmail.com, vinicius.mctf@grad.ufsc.br}
}

\begin{document} 

\maketitle
     
\begin{resumo} 

    \input{sections/00-resumo}

\end{resumo}

%máximo de 10 linhas!

\input{sections/01-introducao}
\input{sections/02-componente}
%\input{sections/03-funcionamento}
\input{sections/04-interface}
%\input{sections/05-aplicacao}
\input{sections/06-testes}
\input{sections/07-conclusao}

\bibliographystyle{sbc}
\bibliography{soii}

\end{document}
